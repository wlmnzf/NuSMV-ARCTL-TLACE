When the \commandopt{int} option is not specified, \nusmv
runs as a batch program, in the style of \smv, performing (some
of) the steps described in previous section in a fixed sequence.
\begin{alltt}
\shellprompt \shelltext{\nusmvtxt [command line options] {\it input-file}} \ret
\end{alltt}
The program described in {\it input-file} is processed, and the
corresponding finite state machine is built.  Then, if \emph{input-file}
contains formulas to verify, their truth in the specified structure is
evaluated. For each formula which is not true a counterexample is
printed.\\
The batch mode can be controlled with
the following command line options:\\
\begin{alltt}
\nusmv [-h | -help] [-v {\it vl}] [-int] [-load {\it script\_file}] [-s]
       [-cpp] [-pre {\it pps}] [-ofm {\it fm\_file}] [-obm {\it bm\_file}]
       [-lp] [-n {\it idx}] [-is] [-ic] [-ils] [-ii] 
       [-ctt] [-f] [-r] [-AG]  [-coi]
       [-i {\it iv\_file}] [-o {\it ov\_file}] [-reorder] [-dynamic] [-m {\it method}]
       [[-mono]|[-thresh {\it cp\_t}]|[-cp {\it cp\_t}]|[-iwls95 {\it cp\_t}]]
       [-noaffinity] [-iwls95preorder]
       [-bmc] [-bmc\_length {\it k}]
       [{\it input-file}]
\end{alltt}
where the meaning of the options is described below. If
{\it input-file} is not provided in batch mode, then the model is read
from standard input.\\

\begin{nusmvTable}

\opt{-help}{ }

\opt{-h}{
\index{ \code{-help}}
\index{ \code{-h}}
Prints the command line help.}

\opt{-v {\it verbose-level}}{
\index{\code{-v} {\it verbose-level}}
Enables printing of additional information on the internal operations of
\nusmv. Setting {\it verbose-level} to \code{1} gives the basic
information. Using this option makes you feel better, since otherwise
the program prints nothing until it finishes, and there is no evidence
that it is doing anything at all. Setting the {\it verbose-level}
higher than 1 enables printing of much extra information.}

\opt{-int}{
\index{ \code{-int}}
Starts interactive shell.}

\opt{-load {\it cmd-file}}{
\index{ \code{-load} {\it cmd-file}}
Interprets \nusmv commands from file {\it cmd-file}.}

\opt{-s} {
Avoids to load the \nusmv commands contained in \code{
\~/.nusmvrc} or in \code{.nusmvrc}  or in 
\code{\$\$\{\stdsyslib\}/master.nusmvrc}.} 
\vindex{\stdsyslib}
\index{master.nusmvrc}
\index{.nusmvrc}
\index{\~/.nusmvrc}

\opt{-cpp}{
\index{\code{-cpp}}
Runs preprocessor on \smv files before any of those specified with the
-pre option.}

\opt{-pre {\it pps}}{
\index{\code{-pre} {\it pps}}
Specifies a list of pre-processors to run (in the order given) on the
input file before it is parsed by \nusmv. Note that if the
\commandopt{cpp} command is used, then the pre-processors specified by
this command will be run after the input file has been pre-processed
by that pre-processor. {\it pps} is either one single pre-processor
name (with or without double quotes) or it is a space-seperated list
of pre-processor names contained within double quotes.}

\opt{-ofm {\it fm\_file}}{
\index{ \code{-ofm} {\it fm\_file}}
prints flattened model to file {\it fn\_file}}

\opt{-obm {\it bm\_file}} {
\index{ \code{-obm} {\it bm\_file}}
Prints boolean model to file {\it bn\_file}}

\opt{-lp}{
\index{\code{-lp}}
Lists all properties in \smv model}

\opt{-n {\it idx}}{
\index{\code{-n} {\it idx}}
Specifies which property of \smv model should be checked}

\opt{-is}{
\index{\code{-is}}
Does not check \code{SPEC}}

\opt{-ic}{
\index{\code{-ic}}
Does not check \code{COMPUTE}}

\opt{-ils}{
\index{\code{-ils}}
Does not check \code{LTLSPEC}}

\opt{-ii}{
\index{ \code{-ii}}
Does not check \code{INVARSPEC}}

\opt{-ctt}{
\index{\code{-ctt}}
Checks whether the transition relation is total.}

\opt{-f}{
\index{ \code{-f}}
Computes the set of reachable states before evaluating CTL expressions.}

\opt{-r}{
\index{ \code{-r}}
Prints the number of reachable states before exiting. If the \commandopt{f}
option is not used, the set of reachable states is computed.}

\opt{-AG}{
\index{ \code{-AG}}
Verifies only AG formulas using an ad hoc algorithm (see documentation for
the \envvar{ag\_only\_search} environment variable).}

\opt{-coi} {
\index{ \code{-coi}}
Enables cone of influence reduction }

\opt{-i {\it iv\_file}} {
\index{ \code{-i} {\it iv\_file}}
Reads the variable ordering from file {\it iv\_file}. }

\opt{-o {\it ov\_file}}{
\index{ \code{-o} {\it ov\_file}}
Reads the variable ordering from file {\it ov\_file}.}

\opt{-reorder}{
\index{ \code{-reorder}}
Enables variable reordering after having checked all the specification if
any.}

\opt{-dynamic}{
\index{ \code{-dynamic}}
Enables dynamic reordering of variables}

\end{nusmvTable}


\begin{nusmvTable}

\opt{-m  {\it method}} {
\index{ \code{-m} {\it method}}
Uses {\it method} when variable ordering is enabled. Possible values for
method are those allowed for the \envvar{reorder\_method} environment
variable (see \sref{Interface to DD package}).}

\opt{-mono}{
\index{ \code{-mono}}
Enables monolithic transition relation}

\opt{-thresh {\it cp\_t}}{ 
\index{ \code{-thresh} {\it cp\_t}}
conjunctive partitioning with threshold of each
partition set to {\it cp\_t} (DEFAULT, with {\it cp\_t}=1000)}

\opt{-cp {\it cp\_t}}{
\index{ \code{-cp} {\it cp\_t}}
DEPRECATED: use \command{thresh} instead.}

\opt{-iwls95 {\it cp\_t}}{
\index{ \code{-iwls95} {\it cp\_t}}
Enables Iwls95 conjunctive partitioning and sets
the threshold of each partition to {\it cp\_t}}

\opt{-noaffinity}{
\index{ \code{-noaffinity}}
Disables affinity clustering }

\opt{-iwls95preoder} {
\index{ \code{-iwls95preorder}}
Enables \Iwls preordering}

\opt{-bmc} {
\index{ \code{-bmc}}
Enables BMC instead of BDD model checking }

\opt{-bmc {\it k}}{
\index{ \code{-bmc} {\it k}}
Sets \envvar{bmc\_length} variable, used by BMC}

\end{nusmvTable}



